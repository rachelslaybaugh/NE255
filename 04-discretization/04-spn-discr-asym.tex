\documentclass[12pt]{article}
\usepackage[top=1in, bottom=1in, left=1in, right=1in]{geometry}

\usepackage{setspace}
\onehalfspacing

\usepackage{amssymb}
%% The amsthm package provides extended theorem environments
\usepackage{amsthm}
\usepackage{epsfig}
\usepackage{times}
\renewcommand{\ttdefault}{cmtt}
\usepackage{amsmath}
\usepackage{graphicx} % for graphics files
\usepackage{tabu}

% Draw figures yourself
\usepackage{tikz} 

% writing elements
\usepackage{mhchem}

\usepackage{paralist}

% The float package HAS to load before hyperref
\usepackage{float} % for psuedocode formatting
\usepackage{xspace}

% from Denovo Methods Manual
\usepackage{mathrsfs}
\usepackage[mathcal]{euscript}
\usepackage{color}
\usepackage{array}
\usepackage{bm}

\usepackage[parfill]{parskip}

% math syntax
\newcommand{\SN}{S$_N$}
\newcommand{\vd}{\bm{\cdot}} % slightly bold vector dot
\newcommand{\grad}{\vec{\nabla}} % gradient
\newcommand{\ud}{\mathop{}\!\mathrm{d}} % upright derivative symbol
\def\bal#1\nal{\begin{align}#1\end{align}}
\def\bala#1\nala{\begin{align*}#1\end{align*}}
\newcommand{\f}{\frac}
\newcommand{\ux}{\ensuremath{\vec{r}}}
\newcommand{\un}{{\bm n}}
\newcommand{\unab}{{\bf \nabla}}
\newcommand{\ep}{\varepsilon}
\newcommand{\uom}{\ensuremath{\hat{\Omega}}}

\newcommand{\nth}{n\ensuremath{^{\text{th}}} }
\newcommand{\ve}[1]{\ensuremath{\mathbf{#1}}}
\newcommand{\Macro}{\ensuremath{\Sigma}}
\newcommand{\rvec}{\ensuremath{\vec{r}}}
\newcommand{\vecr}{\ensuremath{\vec{r}}}
\newcommand{\omvec}{\ensuremath{\hat{\Omega}}}
\newcommand{\vOmega}{\ensuremath{\hat{\Omega}}}
\newcommand{\even}{\ensuremath{\phi^g}}
\newcommand{\odd}{\ensuremath{\vartheta^g}}
\newcommand{\evenp}{\ensuremath{\phi^{g'}}}
\newcommand{\oddp}{\ensuremath{\vartheta^{g'}}}
\newcommand{\Sn}{\ensuremath{S_N} }
\newcommand{\Ye}[2]{\ensuremath{Y^e_{#1}(\vOmega_#2)}}
\newcommand{\sigg}[1]{\ensuremath{\Macro^{gg'}_{s\,#1}}}
\newcommand{\psig}{\ensuremath{\psi^g}}
%---------------------------------------------------------------------------
%---------------------------------------------------------------------------
\begin{document}
\begin{center}
{\bf NE 255, Fa16 \\
Simplified P$_N$ Equations - part 2\\
October 11, 2016}
\end{center}

\setlength{\unitlength}{1in}
\begin{picture}(6,.1) 
\put(0,0) {\line(1,0){6.25}}         
\end{picture}

There are at least two ways that asymptotic analysis can be used to derive the SP$_N$ equations, both dating back to the early 1990's.
(I) Pomraning presented a derivation demonstrating that, if the transport solution is locally 1-D, the SP$_N$ solution will asymptotically agree with the transport solution.
(II) Larsen, Morel, and McGhee derived the SP$_N$ equations for anisotropic scattering that is not highly forward peaked and included energy dependence through the multigroup method.
Neither of these derivations includes asymptotic boundary conditions.

\section*{Asymptotic Derivation of the SP$_N$ Equations}
We will study the second approach for isotropic scattering in the one-speed case.
The derivation when there is anisotropic scattering requires some tensor analysis that is beyond the scope of this class.
The transport equation is written as
\bal\label{1}
&\uom \cdot \unab \psi(\ux,\uom) + \Sigma_t(\ux)\psi(\ux,\uom) = \f{\Sigma_s(\ux)}{4\pi}\phi(\ux) + \frac{Q(\ux)}{4\pi},
\nal
where $\phi = \int_{4\pi}\psi d\Omega=$ P$_0$ moment of $\psi$.
Defining $0<\ep\ll 1$, we scale the parameters in order to consider a \textit{diffusive} system:
\bala
\Sigma_t(\ux) &= \f{\sigma_t(\ux)}{\ep},\\
\Sigma_t(\ux) - \Sigma_s(\ux) &= \Sigma_a(\ux) = \ep\sigma_a(\ux), \\
Q(\ux) &= \ep q(\ux),
\nala
where $\sigma_t$, $\sigma_a$, and $q$ are $O(1)$.
The physics implied by this scaling is as follows:
\begin{enumerate}
\item The system is optically thick ($\Sigma_t >>1$)
\item The rates of absorption and production due to interior sources are comparable and weak: $\Sigma_a = O(\ep)$ and $Q = O(\ep)$
\item The infinite medium solution $\phi = Q/\Sigma_a = q/\sigma_a$ is $O(1)$ \item The diffusion length $L = (3\Sigma_t\Sigma_a)^{-1/2}=(3\sigma_t\sigma_a)^{-1/2}$ is $O(1)$
\item If one introduces this scaling into the standard diffusion approximation, the resulting equation is independent of $\ep$. In other words, the standard diffusion equation is invariant under this scaling. 
\end{enumerate}

Under this scaling, Eq.~\eqref{1} becomes
\bala
\left(1+\f{\ep}{\sigma_t}\uom\cdot\unab\right)\psi = \f{1-\ep^2\sigma_a/\sigma_t}{4\pi}\phi +\f{\ep^2 q}{4\pi\sigma_t}.
\nala
We invert the operator on the left-hand side of this equation to obain and expression for $\psi$ in terms of $\phi$ and $q$:
\bala
\psi = \left(1+\f{\ep}{\sigma_t}\uom\cdot\unab\right)^{-1}\left[\f{1-\ep^2\sigma_a/\sigma_t}{4\pi}\phi +\f{\ep^2 q}{4\pi\sigma_t}\right].
\nala
Since $\ep$ is small, we expand the inverse operator in a power series:
\bal
\psi &= \left(1-\ep\left(\f{1}{\sigma_t}\uom\cdot\unab\right) + \ep^2\left(\f{1}{\sigma_t}\uom\cdot\unab\right)^2-\ep^3\left(\f{1}{\sigma_t}\uom\cdot\unab\right)^3+\ep^4\left(\f{1}{\sigma_t}\uom\cdot\unab\right)^4-\right.\label{2}\\
&\qquad\qquad\left. \ep^5\left(\f{1}{\sigma_t}\uom\cdot\unab\right)^5+\ep^6\left(\f{1}{\sigma_t}\uom\cdot\unab\right)^6+O(\ep^7)\right)\left[\f{1-\ep^2\sigma_a/\sigma_t}{4\pi}\phi +\f{\ep^2 q}{4\pi\sigma_t}\right].\nonumber
\nal
Next, we will integrate Eq.~\eqref{2} upon the unit sphere and divide it by $4\pi$.
We need the following identity:
\bala
\f{1}{4\pi}\int_{4\pi}\left(\f{1}{\sigma_t}\uom\cdot\unab\right)^nd\Omega = 
\f{1+(-1)^n}{2}\f{1}{n+1}\left(\f{1}{\sigma_t}\unab\right)^n,
\nala
for $n=0,1,2,...$ .
Thus, Eq.~\eqref{2} becomes
\bal\label{3}
\f{\phi}{4\pi} &= \left(1+ \f{\ep^2}{3}\left(\f{1}{\sigma_t}\unab\right)^2 + \f{\ep^4}{5}\left(\f{1}{\sigma_t}\unab\right)^4
+ \f{\ep^6}{7}\left(\f{1}{\sigma_t}\unab\right)^6 + O(\ep^8)\right)\left[\f{1-\ep^2\sigma_a/\sigma_t}{4\pi}\phi +\f{\ep^2 q}{4\pi\sigma_t}\right].
\nal
If there are non-vacuum boundary conditions, then extra terms occur in Eq.~\eqref{3}.
However, these are $O(e^{-\rho/\ep})$, where $\rho$ is the optical distance to the boundary.
Thus, in the interior of the system these terms are exponentially small and we will ignore them. 
\newpage

Now we invert the operator on the right-hand side of this equation and once again expand it in a power series:
\bala
\left(1+ \f{\ep^2}{3}\left(\f{1}{\sigma_t}\unab\right)^2 + \f{\ep^4}{5}\left(\f{1}{\sigma_t}\unab\right)^4
+ \f{\ep^6}{7}\left(\f{1}{\sigma_t}\unab\right)^6 + O(\ep^8)\right)^{-1} &= \\
&\hspace{-17em}1 - \left(\f{\ep^2}{3}\left(\f{1}{\sigma_t}\unab\right)^2 + \f{\ep^4}{5}\left(\f{1}{\sigma_t}\unab\right)^4
+ \f{\ep^6}{7}\left(\f{1}{\sigma_t}\unab\right)^6 + O(\ep^8)\right) + \\
& \hspace{-9.8em} \left(\f{\ep^4}{9}\left(\f{1}{\sigma_t}\unab\right)^4 + \f{2\ep^6}{15}\left(\f{1}{\sigma_t}\unab\right)^6 + O(\ep^8)\right) -\\
& \hspace{-3.2em}\left(\f{\ep^6}{27}\left(\f{1}{\sigma_t}\unab\right)^6 + O(\ep^8)\right) =\\
&\hspace{-17em} = 1-\f{\ep^2}{3}\left(\f{1}{\sigma_t}\unab\right)^2 - \f{4\ep^4}{45}\left(\f{1}{\sigma_t}\unab\right)^4 -
\f{44\ep^6}{945}\left(\f{1}{\sigma_t}\unab\right)^6 + O(\ep^8) \,.
\nala
Thus, Eq.~\eqref{3} becomes
\bala
\left(1-\f{\ep^2}{3}\left(\f{1}{\sigma_t}\unab\right)^2 - \f{4\ep^4}{45}\left(\f{1}{\sigma_t}\unab\right)^4 -
\f{44\ep^6}{945}\left(\f{1}{\sigma_t}\unab\right)^6 + O(\ep^8)\right)\phi = 
\left(1-\ep^2\sigma_a/\sigma_t\right)\phi +\f{\ep^2 q}{\sigma_t}\,,
\nala
or
\bal\label{4}
-\sigma_t\left(\f{1}{3}\left(\f{1}{\sigma_t}\unab\right)^2 + \f{4\ep^2}{45}\left(\f{1}{\sigma_t}\unab\right)^4 +
\f{44\ep^4}{945}\left(\f{1}{\sigma_t}\unab\right)^6 + O(\ep^6)\right)\phi +\sigma_a\phi = q\,.
\nal
If we now retain terms of $O(\ep^{2n})$ but discard all higher order terms, we obtain a partial differential equation for $\phi$ of order $2n$.
This equation is an asymptotic approximation to Eq.~\eqref{3}, but it is \textit{not} any of the simplifieded P$_N$ approximations.
To derive these approximations, we must rewrite the equation obtained from Eq.~\eqref{4} in an asymptotically equivalent form as either a single second-order
equation or as a coupled system of second-order equations.
We will do that now for SP$_1$, SP$_2$, and SP$_3$. 

\subsection*{Diffusion Equation (P$_1$)}
We delete terms of $O(\ep^2)$ and higher in Eq.~\eqref{4} to get
\bala
-\unab\cdot\f{1}{3\sigma_t}\unab\phi + \sigma_a\phi = q.
\nala
Multiplying this by $\ep$ and going back to the original unscaled parameters, we get the diffusion equation (P$_1$ or SP$_1$):
\bala
-\unab\cdot\f{1}{3\Sigma_t(\ux)}\unab\phi(\ux) + \Sigma_a(\ux)\phi(\ux) = Q(\ux).
\nala

\subsection*{SP$_2$ Equation}
We delete terms of $O(\ep^4)$ and higher in Eq.~\eqref{4} and rearrange the terms:
\bala
\left(I+\f{4\ep^2}{15}\left(\f{1}{\sigma_t}\unab\right)^2\right)\f{1}{3}\left(\f{1}{\sigma_t}\unab\right)^2\phi = \f{\sigma_a\phi-q}{\sigma_t}.
\nala
Then we operate on this equation by $\left(I-\f{4\ep^2}{15}\left(\f{1}{\sigma_t}\unab\right)^2\right)$ and discard terms of $O(\ep^4)$ to get
\bala
\f{1}{3}\left(\f{1}{\sigma_t}\unab\right)^2\phi = \left(I-\f{4\ep^2}{15}\left(\f{1}{\sigma_t}\unab\right)^2\right)\f{\sigma_a\phi-q}{\sigma_t},
\nala
which simplifies to
\bala
-\unab\cdot\f{1}{3\sigma_t}\unab\left(\phi + \f{4\ep^2}{5}\f{\sigma_a\phi-q}{\sigma_t}\right) + \sigma_a\phi = q.
\nala
Multiplying this by $\ep$ and going back to the original unscaled parameters, we get the simplified P$_2$ equation:
\bala
-\unab\cdot\f{1}{3\Sigma_t(\ux)}\unab\left(\phi(\ux) + \f{4}{5}\f{\Sigma_a(\ux)\phi(\ux)-Q(\ux)}{\Sigma_t(\ux)}\right) + \Sigma_a(\ux)\phi(\ux) = Q(\ux).
\nala

\subsection*{SP$_3$ Equations}
We delete terms of $O(\ep^6)$ and higher in Eq.~\eqref{4} and rearrange the terms:
\bal\label{5}
-\sigma_t\f{1}{3}\left(\f{1}{\sigma_t}\unab\right)^2\left(\phi +\f{4\ep^2}{15}\left(\f{1}{\sigma_t}\unab\right)^2\phi + \f{44\ep^4}{315}\left(\f{1}{\sigma_t}\unab\right)^4\phi\right) + \sigma_a\phi = q.
\nal
Now we define
\bal\label{6}
\phi_2 = \f{2\ep^2}{15}\left(\f{1}{\sigma_t}\unab\right)^2\phi + \f{22\ep^4}{315}\left(\f{1}{\sigma_t}\unab\right)^4\phi = \left(I + \f{11\ep^2}{21}\left(\f{1}{\sigma_t}\unab\right)^2\right)\f{2\ep^2}{15}\left(\f{1}{\sigma_t}\unab\right)^2\phi
\nal
and rewrite Eq.~\eqref{5} as
\bal\label{7}
-\unab\cdot\f{1}{3\sigma_t}\unab\left(\phi + 2\phi_2\right) + \sigma_a\phi = q.
\nal
Operating on Eq.~\eqref{6} by
\[
\left(I - \f{11\ep^2}{21}\left(\f{1}{\sigma_t}\unab\right)^2\right)
\]
and discarding terms of $O(\ep^6)$, we obtain
\bala
\left(I - \f{11\ep^2}{21}\left(\f{1}{\sigma_t}\unab\right)^2\right)\phi_2 = 
\f{2\ep^2}{15}\left(\f{1}{\sigma_t}\unab\right)^2\phi,
\nala
which simplifies to
\bal\label{8}
-\f{1}{\sigma_t}\unab\cdot\f{1}{3\sigma_t}\unab\left(\f{11\ep^2}{7}\phi_2 + \f{2\ep^2}{5}\phi\right) + \phi_2 = 0.
\nal
Multiplying Eq.~\eqref{7} by $\ep$ and going back to the original unscaled parameters, we get
\bala
-\unab\cdot\f{1}{3\Sigma_t(\ux)}\unab\left(\phi(\ux) + 2\phi_2(\ux)\right) + \Sigma_a(\ux)\phi(\ux) = Q(\ux);
\nala
similarly, multiplying Eq.~\eqref{8} by $\sigma_t/\ep$ and going back to the original unscaled parameters, we get
\bala
-\unab\cdot\f{1}{3\Sigma_t(\ux)}\unab\left(\f{11}{7}\phi_2(\ux) + \f{2}{5}\phi(\ux)\right) + \Sigma_t(\ux)\phi_2(\ux) = 0.
\nala
These are the simplified P$_3$ equations for Eq.~\eqref{1}.

NOTE: In the previous class we had obtained the following SP$_3$ equations for a problem with an isotropic source:
\begin{align*}
&-\nabla \cdot \frac{1}{3[\Sigma_t-\Sigma_{s1}]}\nabla\phi_0
-\nabla \cdot \frac{2}{3[\Sigma_t-\Sigma_{s1}]}\nabla\phi_2
+\Sigma_a\phi_0 = s_0\,,\\
&-\nabla \cdot \frac{2}{15[\Sigma_t-\Sigma_{s1}]}\nabla\phi_0
-\nabla \cdot \left(\frac{4}{15[\Sigma_t-\Sigma_{s1}]}+\frac{9}{35[\Sigma_t-\Sigma_{s3}]}\right)\nabla\phi_2
+[\Sigma_t-\Sigma_{s2}]\phi_2 = 0\,.
 \end{align*}
If scattering is isotropic, $\Sigma_{s1} = \Sigma_{s2} = \Sigma_{s3} = 0$, and these equations simplify to
\begin{align*}
&-\nabla \cdot \frac{1}{3\Sigma_t}\nabla\phi_0
-\nabla \cdot \frac{2}{3\Sigma_t}\nabla\phi_2
+\Sigma_a\phi_0 = s_0\,,\\
&-\nabla \cdot \frac{2}{15\Sigma_t}\nabla\phi_0
-\nabla \cdot \left(\frac{4}{15\Sigma_t}+\frac{9}{35\Sigma_t}\right)\nabla\phi_2
+\Sigma_t\phi_2 = 0\,,
 \end{align*}
 reducing to the SP$_3$ equations derived above.
 
 
 
--------------------------

\vfill

These notes are derived from Ryan McClarren's review paper on the SP$_N$ equations: ``Theoretical Aspects of the Simplified P$_n$ Equations'', Transport Theory and Statistical Physics 39: 73--109, 2011; and from the conference paper by Larsen, Morel, and McGhee: ``Asymptotic Derivation of the Simplified P$_N$ Equations'', Proceedings of the ANS - M\&C Topical Meeting in Karlsruhe, Germany (1993).


\end{document}
Contact GitHub 