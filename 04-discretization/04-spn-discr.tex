\documentclass[12pt]{article}
\usepackage[top=1in, bottom=1in, left=1in, right=1in]{geometry}

\usepackage{setspace}
\onehalfspacing

\usepackage{amssymb}
%% The amsthm package provides extended theorem environments
\usepackage{amsthm}
\usepackage{epsfig}
\usepackage{times}
\renewcommand{\ttdefault}{cmtt}
\usepackage{amsmath}
\usepackage{graphicx} % for graphics files
\usepackage{tabu}

% Draw figures yourself
\usepackage{tikz} 

% writing elements
\usepackage{mhchem}

\usepackage{paralist}

% The float package HAS to load before hyperref
\usepackage{float} % for psuedocode formatting
\usepackage{xspace}

% from Denovo Methods Manual
\usepackage{mathrsfs}
\usepackage[mathcal]{euscript}
\usepackage{color}
\usepackage{array}

\usepackage[pdftex]{hyperref}
\usepackage[parfill]{parskip}

% math syntax
\newcommand{\nth}{n\ensuremath{^{\text{th}}} }
\newcommand{\ve}[1]{\ensuremath{\mathbf{#1}}}
\newcommand{\Macro}{\ensuremath{\Sigma}}
\newcommand{\rvec}{\ensuremath{\vec{r}}}
\newcommand{\vecr}{\ensuremath{\vec{r}}}
\newcommand{\omvec}{\ensuremath{\hat{\Omega}}}
\newcommand{\vOmega}{\ensuremath{\hat{\Omega}}}
\newcommand{\even}{\ensuremath{\phi^g}}
\newcommand{\odd}{\ensuremath{\vartheta^g}}
\newcommand{\evenp}{\ensuremath{\phi^{g'}}}
\newcommand{\oddp}{\ensuremath{\vartheta^{g'}}}
\newcommand{\Sn}{\ensuremath{S_N} }
\newcommand{\Ye}[2]{\ensuremath{Y^e_{#1}(\vOmega_#2)}}
\newcommand{\sigg}[1]{\ensuremath{\Macro^{gg'}_{s\,#1}}}
\newcommand{\psig}{\ensuremath{\psi^g}}
%---------------------------------------------------------------------------
%---------------------------------------------------------------------------
\begin{document}
\begin{center}
{\bf NE 255, Fa16 \\
Simplified P$_N$ Equations\\
October 04, 2016}
\end{center}

\setlength{\unitlength}{1in}
\begin{picture}(6,.1) 
\put(0,0) {\line(1,0){6.25}}         
\end{picture}

In slab geometry the P$_N$ equations can be written as a system of 1-D diffusion equations; this is not true in general geometry.
This is the motivation behind the simplified P$_N$ equations: what would happen if the P$_N$ method in general geometry was as nice as it is in slab geometry?

Gelbard introduced the SP$_N$ equations in a series of papers in 1962; however, they were not widely accepted as an approximate transport method because of the lack of a true theoretical foundation.
For approximately 30 years, the SP$_N$ equations were occasionally mentioned
in American Nuclear Society conference talks and brief publications.
It was not until the early 1990's that theoretical work was published demonstrating that the SP$_N$ approximations have a valid mathematical foundation, and can be derived from either an asymptotic or a variational analysis.

\subsection*{``Heuristic'' Derivation of the SP$_N$ Equations}

Consider the planar (slab) geometry P$_N$ equations as before: for $l' = 0, 1, \dots$, we have
\[
\left(\frac{l'+1}{2l'+1}\right)\frac{d}{d x}\phi_{l'+1}(x) + \left(\frac{l'}{2l'+1}\right)\frac{d}{d x}\phi_{l'-1}(x) + \Sigma_t(x) \phi_{l'} = \Sigma_{sl'}(x)\phi_{l'}(x) + s_{l'}(x)\:,
\]
with $\phi_{-1}=0$ and
\[\phi_{N+1} = 0 \quad\text{ or }\quad\frac{d}{dx}\phi_{N+1}=0\,.
\]

The second-order form of the planar geometry P$_1$ equations with Marshak boundary conditions is the diffusion equation
\begin{align*}
&-\frac{d }{dx}D\frac{d \phi_0}{dx} + \Sigma_a(x) \phi_0(x) = s_0(x)\:, 0<x<X\,,\\
&\frac{1}{2}\phi_0(0)-D\frac{d\phi_0}{dx}(0) = 2J^+(0),\\
&\frac{1}{2}\phi_0(X)+D\frac{d\phi_0}{dx}(X) = 2J^-(X),
\end{align*}
where
\[
D = \frac{1}{3\left[\Sigma_t(x) - \Sigma_{s1}(x)\right]}.
\]

This can be generalized to 3-D by making the two \textbf{formal} modifications: 
\begin{enumerate}
\item Replace the 1-D diffusion operator 
\[
\frac{d }{dx}D\frac{d }{dx}
\]
by the 3-D diffusion operator
\[
\nabla\cdot D\nabla \equiv \frac{\partial }{\partial x}D\frac{\partial }{\partial x}+\frac{\partial }{\partial y}D\frac{\partial }{\partial y}+\frac{\partial }{\partial z}D\frac{\partial }{\partial z}\,;
\]
\item In the boundary conditions, replace the derivative terms
\[
\pm\frac{d }{dx}
\]
by the outward normal derivative
\[
\vec{n} \cdot \nabla
\]
\end{enumerate}
Making these formal modifications, we obtain the standard 3-D diffusion (P$_1$) equations
\begin{align*}
&-\nabla\cdot D\nabla\phi_0(\vecr) + \Sigma_a(\vecr) \phi_0(\vecr) = s_0(\vecr)\:, \vecr \in V,\\
&\frac{1}{2}\phi_0(\vecr)+D\vec{n}\cdot\nabla\phi_0 = 2J^-(\vecr), \, \vecr\in \partial V\,.
\end{align*}
These equations obviously reduce to the standard 1-D diffusion equations in planar geometry.

We carry out the same procedure for the general SP$_N$ equations. First, for odd values of $l'$, $\phi_{l'}$ is replaced by a vector:
\[
\phi_{l'} \rightarrow \vec{\phi}_{l'} = (\phi_{l'}^x,\phi_{l'}^y,\phi_{l'}^z)^t\,.
\]
Then, in the even $l'$ equations the derivative in $x$ is replaced by a divergence: 
\[
\frac{d}{dx} \rightarrow \nabla \cdot\,;
\]
and in the odd $l'$ equations the x derivative is changed to a gradient: 
\[
\frac{d}{dx} \rightarrow \nabla
\]
This allows us to write the first-order form of the SP$_N$ equations as 
\begin{align*}
&\nabla \cdot \vec{\phi}_1 + \Sigma_a\phi_0 = s_0\,,&&\\
&\left(\frac{l'+1}{2l'+1}\right)\nabla\phi_{l'+1} + \left(\frac{l'}{2l'+1}\right)\nabla\phi_{l'-1} + \Sigma_t \vec{\phi}_{l'} = \Sigma_{sl'}\vec{\phi}_{l'} + s_{l'}\:, & \text{for odd $l'$,}&\\
&\left(\frac{l'+1}{2l'+1}\right)\nabla\cdot\vec{\phi}_{l'+1} + \left(\frac{l'}{2l'+1}\right)\nabla\cdot\vec{\phi}_{l'-1} + \Sigma_t \phi_{l'} = \Sigma_{sl'}\phi_{l'} + s_{l'}\:, &\text{for even $l'>0$.}&
\end{align*}
The boundary conditions for the SP$_N$ equations can be obtained from the P$_N$ Marshak boundary conditions by replacing $\phi_{l'}$ with the SP$_N$ unknowns and $\mu$ with $\vec{n}\cdot\omvec$, where $\vec{n}$ is the unit inward normal to the boundary.

\subsection*{The SP$_3$ Equations}

Assuming an isotropic source, the SP$_3$ equations in their first-order form are
\begin{align*}
\nabla\cdot\vec{\phi}_1 + \Sigma_a\phi_0 &= s_0\,,\\
\frac{1}{3}\nabla\phi_0 + \frac{2}{3}\nabla\phi_2 + [\Sigma_t-\Sigma_{s1}]\vec{\phi}_1 &= 0\,,\\
\frac{2}{5}\nabla\cdot\vec{\phi}_1 + \frac{3}{5}\nabla\cdot\vec{\phi}_3 + [\Sigma_t-\Sigma_{s2}]\phi_2 &= 0\,,\\
\frac{3}{7}\nabla\phi_2 + [\Sigma_t-\Sigma_{s3}]\vec{\phi}_3 &= 0\,.
\end{align*}
We can rewrite them in their second-order form by using the relation
\[
\vec{\phi}_{l'} = -\frac{1}{\Sigma_t-\Sigma_{sl'}}\left(\frac{l'}{2l'+1}\nabla\phi_{l'-1}+\frac{l'+1}{2l'+1}\nabla\phi_{l'+1}\right)\,,
\]
yielding
\begin{align*}
&-\nabla \cdot \frac{1}{3[\Sigma_t-\Sigma_{s1}]}\nabla\phi_0
-\nabla \cdot \frac{2}{3[\Sigma_t-\Sigma_{s1}]}\nabla\phi_2
+\Sigma_a\phi_0 = s_0\,,\\
&-\nabla \cdot \frac{2}{15[\Sigma_t-\Sigma_{s1}]}\nabla\phi_0
-\nabla \cdot \left(\frac{4}{15[\Sigma_t-\Sigma_{s1}]}+\frac{9}{35[\Sigma_t-\Sigma_{s3}]}\right)\nabla\phi_2
+[\Sigma_t-\Sigma_{s2}]\phi_2 = 0\,.
 \end{align*}
The second-order form is useful because it makes the SP$_N$ equations look like a set of coupled diffusion equations. 

The SP$_3$ equations can be manipulated into a form that resembles a two group diffusion equation by defining $\hat\phi_0 = \phi_0+2\phi_2$.
Using this new variable, we can write
\begin{align*}
&-\nabla \cdot \frac{1}{3[\Sigma_t-\Sigma_{s1}]}\nabla\hat\phi_0
+\Sigma_a\hat\phi_0 = 2\Sigma_a\phi_2 + s_0\,,\\
&-\nabla \cdot \frac{9}{35[\Sigma_t-\Sigma_{s3}]}\nabla\phi_2
+ \left([\Sigma_t-\Sigma_{s2}]+\frac{4}{5}\Sigma_a\right)\phi_2
+ = \frac{2}{5}\left[\Sigma_a\hat\phi_0-s_0\right]\,.
 \end{align*}
These equations can be solved with a two-group diffusion code by properly setting the diffusion coefficients and cross-sections or with a one-group diffusion code utilizing an iteration strategy for the coupling terms (FLIP).

\subsection*{General Properties of the SP$_N$ Equations}

The SP$_N$ equations can be understood as a ``super'' diffusion theory.
The structure of the SP$_N$ equations is that of a coupled system of diffusion equations, and the class of problems for which the SP$_N$ equations are accurate encompasses the class of problems for which diffusion theory is accurate.

\begin{enumerate}
\item In 1-D planar geometry, SP$_N$ and P$_N$ are identical
\item In multidimensional problems, SP$_N$ form a system of (N + 1) equations; P$_N$ form a much larger system of (N + 1)$^2$ equations
\item The SP$_N$ equations have the same ``diffusion'' (elliptic) structure as the P$_1$ equations; the P$_N$ equations have a more complicated (hyperbolic) mathematical structure.
\item The above derivation of the SP$_N$ equations assumes as its starting point a 1-group transport problem.
However, applying the same procedures to a multigroup transport problem is straightforward.
The only complication is that the diffusion coefficients can become non-diagonal matrices.
Thus, unlike standard multigroup diffusion theory (but like standard multigroup P$_1$ theory), the multigroup SP$_N$ equations generally have non-diagonal matrix diffusion coefficients.
\item In principle, the 2-D or 3-D SP$_N$ equations can be implemented in a 2-D or 3-D diffusion code without fundamentally rewriting the code.
This is not the case for the P$_N$ equations.
\item The SP$_N$ equations contain more ``transport physics'' than the diffusion equations.
For this reason, solutions of the SP$_N$ equations can contain boundary layers that are not present in P$_1$ solutions.
In order to properly resolve these boundary layers, it may be necessary to use a finer spatial grid for the SP$_N$ equations than for the diffusion equation.
Alternatively, the use of nodal methods with extra expansion terms capable of expressing the boundary layer effects may be required.
\item The multigroup SP$_3$ equations are about twice as costly to solve as the multigroup P$_1$ equations.
However, SP$_3$ solutions are usually much more accurate (transport-like) than P$_1$ solutions.
\item In the limit as $N\rightarrow\infty$, the P$_N$ solutions converge to the transport solution.
\item In the limit as $N\rightarrow\infty$, the SP$_N$ solutions do not generally converge to the transport solution--unless the underlying problem is 1-D.
Therefore, high-order SP$_N$ equations cannot be used to obtain arbitrarily accurate solutions of neutron transport problems in 2 or 3 dimensions.
\item For 3-D problems, the system of P$_N$ equations is much more complicated in structure and greater in number than the system of SP$_N$ equations.
Also, for problems having 1-D symmetry, the P$_N$ and SP$_N$ equations become identical.
For these reasons, it is widely believed that the 3-D SP$_N$ equations can be derived by discarding the proper terms (and equations) from the 3-D P$_N$ equations. However, this has never been shown.
In fact, the precise relationship between the 3-D P$_N$ and the 3-D SP$_N$ equations is not known.
\item For problems in which the P$_1$ solution is accurate, the SP$_3$ solution is generally much more accurate.
As problems become less ``diffusive'' (absorption, streaming, or leakage become increasingly important), the P$_1$ and SP$_3$ solutions both degrade in accuracy.
However, the P$_1$ solutions degrade more rapidly, and the SP$_3$ solutions can
remain accurate well into the range in which P$_1$ solutions are not accurate.
When the problem becomes sufficiently ``difficult'', the P$_1$ and SP$_N$ solutions both become inaccurate (see figure in the next page).
\end{enumerate}

\newpage

\begin{center}
\includegraphics[scale=0.7]{fig_spn}
\end{center}

This figure shows the (qualitative) range of validity of the SP$_N$ equations.
The amounts of absorption and streaming/leakage are indicated on arbitrary scales ranging from 0 to 1.
In region $a$, where streaming, leakage, and absorption are weak, the P$_1$ and all SP$_N$ solutions are accurate.
As absorption or streaming increase (region $b$), P$_1$ becomes inaccurate but SP$_N$ with $N \geq 3$ is still accurate.
As absorption or streaming increase further (region $c$), P$_1$ and SP$_3$ are inaccurate but SP$_N$ with $N \geq 5$ is still accurate.
In region $d$, no SP$_N$ solution is accurate.

--------------------------

\vfill

These notes are derived from Edward Larsen's class notes for NE 644 at the University of Michigan, and from Ryan McClarren's review paper on the SP$_N$ equations: ``Theoretical Aspects of the Simplified P$_n$ Equations'', Transport Theory and Statistical Physics 39: 73--109, 2011.


\end{document}
Contact GitHub 