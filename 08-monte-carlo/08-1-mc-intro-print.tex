%%%%%%%%%%%%%%%%%%%%%%%%%%%%%%%%%%%%%%%%%%%%%%%%%%%%%%%%%%%%
%%  NE 255
%%

\documentclass[xcolor=x11names,compress]{beamer}

\definecolor{CoolBlack}{rgb}{0.0, 0.18, 0.39}
%% General document %%%%%%%%%%%%%%%%%%%%%%%%%%%%%%%%%%
\usepackage{graphicx}
\usepackage{tikz}
\usetikzlibrary{decorations.fractals}
\usepackage{hyperref}
%%%%%%%%%%%%%%%%%%%%%%%%%%%%%%%%%%%%%%%%%%%%%%%%%%%%%%

%% Beamer Layout %%%%%%%%%%%%%%%%%%%%%%%%%%%%%%%%%%
\useoutertheme[subsection=false,shadow]{miniframes}
\useinnertheme{default}
\usefonttheme{serif}
\usepackage{palatino}
\usepackage{tabu}
\usepackage{ulem}
% Links
\usepackage{hyperref}
\definecolor{links}{HTML}{003262}
\hypersetup{colorlinks,linkcolor=,urlcolor=links}

% addition of color
\usepackage{xcolor}
\definecolor{CoolBlack}{rgb}{0.0, 0.18, 0.39}
\definecolor{byellow}{rgb}{0.55037, 0.38821, 0.06142}
\definecolor{dgreen}{rgb}{0.,0.6,0.}
\definecolor{RawSienna}{cmyk}{0,0.72,1,0.45}
\definecolor{forestgreen(web)}{rgb}{0.13, 0.55, 0.13}
\definecolor{cardinal}{rgb}{0.77, 0.12, 0.23}

\setbeamerfont{title like}{shape=\scshape}
\setbeamerfont{frametitle}{shape=\scshape}

\setbeamercolor*{lower separation line head}{bg=CoolBlack}
\setbeamercolor*{normal text}{fg=black,bg=white}
\setbeamercolor*{alerted text}{fg=dgreen} % just testing; I think this looks better
\setbeamercolor*{example text}{fg=black}
\setbeamercolor*{structure}{fg=black}

\setbeamercolor*{palette tertiary}{fg=black,bg=black!10}
\setbeamercolor*{palette quaternary}{fg=black,bg=black!10}

% Margins
\usepackage{changepage}

\mode<presentation>
{
  \definecolor{berkeleyblue}{HTML}{003262}
  \definecolor{berkeleygold}{HTML}{FDB515}
  \usetheme{Boadilla}      % or try Darmstadt, Madrid, Warsaw, Boadilla...
  %\usecolortheme{dove} % or try albatross, beaver, crane, ...
  \setbeamercolor{structure}{fg=berkeleyblue,bg=berkeleygold}
  \setbeamercolor{palette primary}{bg=berkeleyblue,fg=white} % changed this
  \setbeamercolor{palette secondary}{fg=berkeleyblue,bg=berkeleygold} % changed this
  \setbeamercolor{palette tertiary}{bg=berkeleyblue,fg=white} % changed this
  \usefonttheme{structurebold}  % or try serif, structurebold, ...
  \useinnertheme{circles}
  \setbeamertemplate{navigation symbols}{}
  \setbeamertemplate{caption}[numbered]
  \usebackgroundtemplate{}
}
%---

\renewcommand{\(}{\begin{columns}}
\renewcommand{\)}{\end{columns}}
\newcommand{\<}[1]{\begin{column}{#1}}
\renewcommand{\>}{\end{column}}

% adding slide numbers
\addtobeamertemplate{navigation symbols}{}{%
    \usebeamerfont{footline}%
    \usebeamercolor[fg]{footline}%
    \hspace{1em}%
    \insertframenumber/\inserttotalframenumber
}

% equation stuff
\newcommand{\Macro}{\ensuremath{\Sigma}}
\newcommand{\Sn}{\ensuremath{S_N} }
\newcommand{\vOmega}{\ensuremath{\hat{\Omega}}}
\usepackage{mathrsfs}
\usepackage[mathcal]{euscript}
\usepackage{amssymb}
\usepackage{amsthm}
\usepackage{epsfig}
\usepackage{amsmath}
%%%%%%%%%%%%%%%%%%%%%%%%%%%%%%%%%%%%%%%%%%%%%%%%%%
% title stuff for footer
\title{NE 255}
\author{R.\ N.\ Slaybaugh}
\date{November 17, 2016}

\begin{document}

%%%%%%%%%%%%%%%%%%%%%%%%%%%%%%%%%%%%%%%%%%%%%%%%%%%%%%
%%%%%%%%%%%%%%%%%%%%%%%%%%%%%%%%%%%%%%%%%%%%%%%%%%%%%%
\begin{frame}
\title{NE 255\\Numerical Simulations in Radiation Transport}
\subtitle{Introduction to Monte Carlo}
\titlepage
\end{frame}

%%%%%%%%%%%%%%%%%%%%%%%%%%%%%%%%%%%%%%%%%%%%%%%%%%%%%%
%%%%%%%%%%%%%%%%%%%%%%%%%%%%%%%%%%%%%%%%%%%%%%%%%%%%%%
\begin{frame}{Learning Objectives}

\begin{columns}
  \begin{column}{0.5\textwidth}
    \begin{enumerate}
    \item Define Monte Carlo simulation
    \item Justify the choice of Monte Carlo for radiation transport
    \item Understand the mathematical validity of Monte Carlo for radiation
    \item Understand the major components of Monte Carlo methods transport
    \end{enumerate}
  \end{column}
  \begin{column}{0.5\textwidth}
  	\begin{figure}
  	\begin{center}
  		\includegraphics[height=2in,clip]{fig/atr-model}
  		\caption{ATR reactor geometry}
  		%http://www.ornl.gov/science-discovery/more-science/mathematics/nuclear-systems-modeling-and-simulation
	\end{center}
  	\end{figure}
  \end{column}
\end{columns}

Notes derived from Jasmina Vujic and Paul Wilson
\end{frame}

%Patterson and Henesey; computer architecture bible; written at berkeley


%%%%%%%%%%%%%%%%%%%%%%%%%%%%%%%%%%%%%%%%%%%%%%%%%%%%%%
%%%%%%%%%%%%%%%%%%%%%%%%%%%%%%%%%%%%%%%%%%%%%%%%%%%%%%
\begin{frame}{What is Monte Carlo?}

  \begin{itemize}
  \item The use of \textit{random processes} to determine a 
        \textit{statistically-expected} solution to a problem
  \vspace*{1em}
  \item Random processes can fulfill two roles:
  \begin{itemize}
    \item Statistical approximation to \alert{mathematical equations}
    \item Statistical approximations to \alert{physical processes}
  \end{itemize}   
 \vspace*{1em} 
  \item Construct a random process for a problem, 
  \item Carry out a numerical simulation by N-fold sampling from a random \# sequence
\end{itemize}
\end{frame}

%%%%%%%%%%%%%%%%%%%%%%%%%%%%%%%%%%%%%%%%%%%%%%%%%%%%%%
%%%%%%%%%%%%%%%%%%%%%%%%%%%%%%%%%%%%%%%%%%%%%%%%%%%%%%
\begin{frame}{Evaluate $\pi$ by Random Sampling}

\begin{columns}
  \begin{column}{0.45\textwidth}
  	\begin{figure}
  	\begin{center}
  		\includegraphics[height=2in,clip]{fig/pi-circle}
	\end{center}
  	\end{figure}
  \end{column}
  \begin{column}{0.55\textwidth}
    \begin{itemize}
    \item Area of square, $A_s= 4$
    \item Area of circle, $A_c = \pi$
    \item Fraction of random points \\in circle
    \[p = \frac{A_c}{A_s} = \frac{\pi}{4}\]
    \item Random points = $N$
    \item Random points in circle  = $N_c$, $\therefore$
    \[p = \frac{N_c}{N}\:; \quad \pi = \frac{4N_c}{N}\]
    \end{itemize}
  \end{column}
\end{columns}
\end{frame}


%%%%%%%%%%%%%%%%%%%%%%%%%%%%%%%%%%%%%%%%%%%%%%%%%%%%%%
%%%%%%%%%%%%%%%%%%%%%%%%%%%%%%%%%%%%%%%%%%%%%%%%%%%%%%
\begin{frame}{Evaluate $\pi$ by Random Sampling (math)}

\begin{columns}
  \begin{column}{0.3\textwidth}
  	\begin{figure}
  	\begin{center}
  		\includegraphics[height=0.75in,clip]{fig/quarter-circle}
	\end{center}
  	\end{figure}
  \end{column}
  \begin{column}{0.7\textwidth}
    \begin{equation}
      g(x) = \sqrt{1 - x^2}	\qquad G = \int_0^1 g(x)dx = \frac{\pi}{4} \nonumber
    \end{equation}
  \end{column}
\end{columns}

\begin{equation}
  G = \int_0^1 g(x)dx = (1-0)\overline{g(x)} \nonumber
\end{equation}
Determine $\overline{g(x)}$ by random sampling:\\
\vspace*{0.5em}
\hspace*{2 em}for $k = 1, \dots, N$, choose $\hat{x}_k$ randomly on the interval $(0,1)$,
\[ \overline{g(x)} \equiv \frac{1}{N}\sum_{k=1}^N g(\hat{x}_k) = \frac{1}{N}\sqrt{1 - \hat{x}_k^2}\]

\end{frame}


%%%%%%%%%%%%%%%%%%%%%%%%%%%%%%%%%%%%%%%%%%%%%%%%%%%%%%
%%%%%%%%%%%%%%%%%%%%%%%%%%%%%%%%%%%%%%%%%%%%%%%%%%%%%%
\begin{frame}{Evaluate $\pi$ by Random Sampling (physics)}

\begin{columns}
  \begin{column}{0.3\textwidth}
  	\begin{figure}
  	\begin{center}
  		\includegraphics[height=0.75in,clip]{fig/quarter-circle}
	\end{center}
  	\end{figure}
  \end{column}
  \begin{column}{0.7\textwidth}
    \begin{equation}
      g(x) = \sqrt{1 - x^2}	\qquad G = \int_0^1 g(x)dx = \frac{\pi}{4} \nonumber
    \end{equation}
  \end{column}
\end{columns}

\vspace*{0.5em}
G = area under curve, \\
\hspace*{0.75 em} = fraction of unit square under curve\\
\vspace*{0.5em}
\hspace*{2 em}for $k = 1, \dots, N$, chose $\hat{x}_k, \hat{y}_k$ randomly on the interval $[0,1]$,\\
\vspace*{0.5em}
$m_N$ = $\#$ of times in $N$ trials that $\hat{x}_k^2 + \hat{y}_k^2 \leq 1$,
\[G = \frac{m_N}{N}\]

\end{frame}

%%%%%%%%%%%%%%%%%%%%%%%%%%%%%%%%%%%%%%%%%%%%%%%%%%%%%%
%%%%%%%%%%%%%%%%%%%%%%%%%%%%%%%%%%%%%%%%%%%%%%%%%%%%%%
\begin{frame}{Manhattan Project}

\begin{columns}
  \begin{column}{0.5\textwidth}
    \begin{itemize}
    \item The first human engineered nuclear detonation, 
    the Trinity Test in New Mexico.
    \item Active: 1942--1945
    \item Branch: U.S.\ Army Corps of Engineers
    \item Monte Carlo Pioneers:
    \begin{itemize}
      \item Enrico Fermi,
      \item Stanislaw Ulam,
      \item John von Neumann, 
      \item Robert Richtmeyer, 
      \item Nicholas Metropolis
    \end{itemize}
    \end{itemize}
  \end{column}
  \begin{column}{0.5\textwidth}
  	\begin{figure}
  	\begin{center}
  		\includegraphics[height=1.75in,clip]{fig/OppieNeumannMANIAC}
  		\caption{Oppenheimer, von Neumann, MANIAC}
  		%http://www.atomicheritage.org/history/computing-and-manhattan-project
	\end{center}
  	\end{figure}
  \end{column}
\end{columns}

Nicholas Metropolis, S.\ Ulam. ''The Monte Carlo Method," J\textit{ournal of the American Statistical Association}, \textbf{44}, No.\ 247, 335-341 (Sep. 1949).
\end{frame}


%%%%%%%%%%%%%%%%%%%%%%%%%%%%%%%%%%%%%%%%%%%%%%%%%%%%%%
%%%%%%%%%%%%%%%%%%%%%%%%%%%%%%%%%%%%%%%%%%%%%%%%%%%%%%
\begin{frame}{General Purpose MC Codes}

\begin{itemize}
\item \alert{MCNP}: developed at LANL, distributed via RSICC, \href{http://rsicc.ornl.gov}{http://rsicc.ornl.gov}
%
%\item PENELOPE: developed and maintained at U Barcelona, distributed via the Nuclear Energy Agency (http://www.nea.fr/abs/html/nea-1525.html

\item \alert{Geant4}: developed by a large collaboration in the HEP community, \href{ http://geant4.web.cern.ch/geant4/}{http://geant4.web.cern.ch/geant4/}

\item \alert{EGSnrc}: developed at NRC (Canada), \href{http://www.irs.inms.nrc.ca/EGSnrc/EGSnrc.html}{http://www.irs.inms.nrc.ca/EGSnrc/EGSnrc.html}

\item \alert{SERPENT}: Developed by Dr. Jaakko Leppanen, VTT, Finland, \href{ http://montecarlo.vtt.fi/}{http://montecarlo.vtt.fi/}

\item \alert{Shift}: developed at ORNL, distributed via RSICC, \href{http://rsicc.ornl.gov}{http://rsicc.ornl.gov}

\item \alert{Mercury}: developed at LLNL, \href{https://wci.llnl.gov/simulation/computer-codes/mercury}{https://wci.llnl.gov/simulation/computer-codes/mercury}

\end{itemize}

\end{frame}




%%%%%%%%%%%%%%%%%%%%%%%%%%%%%%%%%%%%%%%%%%%%%%%%%%%%%%
%%%%%%%%%%%%%%%%%%%%%%%%%%%%%%%%%%%%%%%%%%%%%%%%%%%%%%
\begin{frame}{Why/When Monte Carlo?}

\begin{itemize}
\item Applications that are mathematically equivalent to \textit{integration over many dimensions}
\vspace*{0.25 em}
\begin{itemize}
\item Analytic integration may be impossible
\vspace*{0.25 em}
\item Deterministic numerical integration may be slow and/or require error prone approximations
\end{itemize} 
\vspace*{0.5 em}

\item However, statistically accurate results can require \textbf{significant computer time}
\item Fortunately, Monte Carlo and parallel computing go well together
\item and we also have Variance Reduction methods
\end{itemize}

\end{frame}


%%%%%%%%%%%%%%%%%%%%%%%%%%%%%%%%%%%%%%%%%%%%%%%%%%%%%%
%%%%%%%%%%%%%%%%%%%%%%%%%%%%%%%%%%%%%%%%%%%%%%%%%%%%%%
\begin{frame}{What is MC Radiation Transport?}

Simulate many independent particles in a system
\begin{itemize}
\item Treat each physical process as a
\textit{probabilistic process}

\item \textit{Randomly sample} each process using an
independent stream of random numbers

\item Follow each particle from birth until it no
longer matters

\item Accumulate the contributions of each
particle to find the statistically-expected
mean behavior and variance
\end{itemize}

\end{frame}


%%%%%%%%%%%%%%%%%%%%%%%%%%%%%%%%%%%%%%%%%%%%%%%%%%%%%%
%%%%%%%%%%%%%%%%%%%%%%%%%%%%%%%%%%%%%%%%%%%%%%%%%%%%%%
\begin{frame}{Mathematical Validity}

\begin{itemize}
\item Consider particles with a phase space
describing position, $\vec{r}$, and velocity, $\vec{v}$
\vspace*{0.5 em}
\item A neutral particle can be transmitted
from one position to another at a
constant velocity
\[T(\vec{r}\:' \rightarrow \vec{r}, \vec{v})\]

\item A particle can undergo a collision at a
single position that changes its velocity
\[C(\vec{r}, \vec{v}\:' \rightarrow \vec{v})\]
\end{itemize}

\end{frame}


%%%%%%%%%%%%%%%%%%%%%%%%%%%%%%%%%%%%%%%%%%%%%%%%%%%%%%
%%%%%%%%%%%%%%%%%%%%%%%%%%%%%%%%%%%%%%%%%%%%%%%%%%%%%%
\begin{frame}{Contributions After 0 Collisions}

\begin{itemize}
\item Consider a particle born from a source
described by 
\[Q(\vec{r}\:', \vec{v}\:')\]
\item This particle will contribute to the flux at $(\vec{r}, \vec{v})$ before any collisions
\end{itemize}

\[\psi_0(\vec{r}, \vec{v}) = \int_{\vec{r}\:'} Q(\vec{r}\:', \vec{v}\:')T(\vec{r}\:' \rightarrow \vec{r}, \vec{v}) d\vec{r}\:'\]

\end{frame}


%%%%%%%%%%%%%%%%%%%%%%%%%%%%%%%%%%%%%%%%%%%%%%%%%%%%%%
%%%%%%%%%%%%%%%%%%%%%%%%%%%%%%%%%%%%%%%%%%%%%%%%%%%%%%
\begin{frame}{Contributions After 1 Collision}

\begin{itemize}
\item The uncollided particles, $\psi_0(\vec{r}\:', \vec{v}\:')$, could undergo 1 \textcolor{cardinal}{collision} and then be \alert{transmitted} to the point $(\vec{r}, \vec{v})$
\end{itemize}

\[\psi_1(\vec{r}, \vec{v}) = \underbrace{\int_{\vec{r}\:'} \biggl[ \underbrace{\int_{\vec{v}\:'} \psi_0(\vec{r}\:', \vec{v}\:') C(\vec{r}\:', \vec{v}\:' \rightarrow \vec{v})  d\vec{v}\:'}_{\textcolor{cardinal}{collision}}\biggr] T(\vec{r}\:' \rightarrow \vec{r}, \vec{v}) d\vec{r}\:'}_{\alert{transmission}}\]

\end{frame}


%%%%%%%%%%%%%%%%%%%%%%%%%%%%%%%%%%%%%%%%%%%%%%%%%%%%%%
%%%%%%%%%%%%%%%%%%%%%%%%%%%%%%%%%%%%%%%%%%%%%%%%%%%%%%
\begin{frame}{Contributions After $k$ Collisions}

\begin{itemize}
\item Particles that have undergone $k$ collisions, $\psi_k(\vec{r}\:', \vec{v}\:')$, could undergo another \textcolor{cardinal}{collision} and then be \alert{transmitted} to the point $(\vec{r}, \vec{v})$ 
\end{itemize}

\[\psi_{k+1}(\vec{r}, \vec{v}) = \underbrace{\int_{\vec{r}\:'} \biggl[ \underbrace{\int_{\vec{v}\:'} \psi_k(\vec{r}\:', \vec{v}\:') C(\vec{r}\:', \vec{v}\:' \rightarrow \vec{v})  d\vec{v}\:'}_{\textcolor{cardinal}{collision}}\biggr] T(\vec{r}\:' \rightarrow \vec{r}, \vec{v}) d\vec{r}\:'}_{\alert{transmission}}\]

\end{frame}



%%%%%%%%%%%%%%%%%%%%%%%%%%%%%%%%%%%%%%%%%%%%%%%%%%%%%%
%%%%%%%%%%%%%%%%%%%%%%%%%%%%%%%%%%%%%%%%%%%%%%%%%%%%%%
\begin{frame}{Combine Collision and Transmission Kernels}

\begin{align*}
\vec{p} &= (\vec{r}, \vec{v}) \quad \text{and}\\
%
R(\vec{p}\:'\rightarrow\vec{p}) &\equiv C(\vec{r}\:', \vec{v}\:' \rightarrow \vec{v}) T(\vec{r}\:' \rightarrow \vec{r}, \vec{v}) \\
&\\
%
\psi_{k+1}(\vec{r}, \vec{v}) = \int_{\vec{p}_k} \psi_k(\vec{p}_k) &R(\vec{p}_k \rightarrow \vec{p}_{k+1}) d\vec{p}_k \\
%
&\\
\psi_{k+1}(\vec{r}, \vec{v}) = \int_{\vec{p}_k} \biggl[ \int_{\vec{p}_{k-1}}  &\psi_{k-1}(\vec{p}_{k-1}) R(\vec{p}_{k-1} \rightarrow \vec{p}_{k}) d\vec{p}_{k-1} \biggr] R(\vec{p}_k \rightarrow \vec{p}_{k+1}) d\vec{p}_k \\
&\text{\dots and so on \dots}
&\\
\psi_{k+1}(\vec{r}, \vec{v}) = \int_{\vec{p}_k} &\int_{\vec{p}_{k-1}} \cdots \int_{\vec{p}_0} \psi_{0}(\vec{p}_{0}) R(\vec{p}_{0} \rightarrow \vec{p}_{1}) d\vec{p}_{0} \cdots \\
&\psi_{k-1}(\vec{p}_{k-1}) R(\vec{p}_{k-1} \rightarrow \vec{p}_{k}) d\vec{p}_{k-1} R(\vec{p}_k \rightarrow \vec{p}_{k+1}) d\vec{p}_k \\
\end{align*}

\end{frame}


%%%%%%%%%%%%%%%%%%%%%%%%%%%%%%%%%%%%%%%%%%%%%%%%%%%%%%
%%%%%%%%%%%%%%%%%%%%%%%%%%%%%%%%%%%%%%%%%%%%%%%%%%%%%%
\begin{frame}{Sum Over All Collisions}

\[\psi(\vec{p}) = \sum_{k=0}^{\infty} \psi_k(\vec{p})\]

\vspace*{1 em}
Arriving at the \textit{integral form} of the transport equation

\[\psi(\vec{r}, \vec{v}) = \int_{\vec{r}\:'} \biggl[ \int_{\vec{v}\:'} \psi(\vec{r}\:', \vec{v}\:') C(\vec{r}\:', \vec{v}\:' \rightarrow \vec{v})  d\vec{v}\:'\biggr] T(\vec{r}\:' \rightarrow \vec{r}, \vec{v}) d\vec{r}\:'\]

\end{frame}


%%%%%%%%%%%%%%%%%%%%%%%%%%%%%%%%%%%%%%%%%%%%%%%%%%%%%%
%%%%%%%%%%%%%%%%%%%%%%%%%%%%%%%%%%%%%%%%%%%%%%%%%%%%%%
\begin{frame}{Mathematical Validity}

\begin{align*}
\Psi_{k}(\vec{p}) = \int\int &\cdots \int \Psi_{0}(\vec{p}_{0}) R(\vec{p}_{0} \rightarrow \vec{p}_{1})R(\vec{p}_{1} \rightarrow \vec{p}_{2}) \\&\cdots R(\vec{p}_{k-1} \rightarrow \vec{p}_{k}) d\vec{p}_{0} d\vec{p}_{1} \cdots d\vec{p}_{k-1}
\end{align*}

\begin{itemize}
\item Integration over many variables
\hspace*{0.5 em}
\item Generate a ``history"\\
(sequence of states $\vec{p}_0, \vec{p}_1 , \dots, \vec{p}_k$)
\begin{itemize}
 \item Randomly sample from source: $\Psi_0 (\vec{p}_0)$
 \item Randomly sample for each of $k$ transitions: $R(\vec{p}_{k-1} \rightarrow \vec{p}_{k})$
\end{itemize}

\item Average for result $A$ by averaging of $M$ histories
\end{itemize}

\[\left\langle A \right\rangle = \int A(\vec{p})\Psi(\vec{p}) d\vec{p} 
= \frac{1}{M} \sum_{m=1}^M \biggl[ \sum_{k=1}^{\infty} A(\vec{p}_{k,m}) \Psi(\vec{p}_{k,m}) \biggr] \]

\end{frame}


%%%%%%%%%%%%%%%%%%%%%%%%%%%%%%%%%%%%%%%%%%%%%%%%%%%%%%
%%%%%%%%%%%%%%%%%%%%%%%%%%%%%%%%%%%%%%%%%%%%%%%%%%%%%%
\begin{frame}{Can Model Very Complex Things}

  	\begin{figure}
  	\begin{center}
  		\includegraphics[height=3in,clip]{fig/pbmr}
	\end{center}
  	\end{figure}

\end{frame}

%%%%%%%%%%%%%%%%%%%%%%%%%%%%%%%%%%%%%%%%%%%%%%%%%%%%%%
%%%%%%%%%%%%%%%%%%%%%%%%%%%%%%%%%%%%%%%%%%%%%%%%%%%%%%
\begin{frame}{Major Components of MC Algorithm}

\begin{itemize}
  \item \textbf{PDFs}: the physical/mathematical system must be described by a set of pdfs.
  % That's next class
  
  \item \textbf{Random number generator}: a source of random \#s uniformly distributed on the unit interval.
  % We're going to skip this
  
  \item \textbf{Sampling rule}: prescription for sampling the pdf (given having random \#s)
  % That's also next class
  
  \item \textbf{Scoring}: the outcomes must be accumulated/\underline{tallied} for quantities of interest
  % we might cover it
  
  \item \textbf{Error estimation}: an estimate of the statistical error (\underline{variance}) of the solution
  % This class!
    
  \item \textbf{Variance Reduction}: methods for reducing the variance and computation time simultaneously
  % briefly this class, maybe more at the end
    
  \item \textbf{Parallelization}: efficient use of computers
  % probably not
\end{itemize}
\end{frame}



%%%%%%%%%%%%%%%%%%%%%%%%%%%%%%%%%%%%%%%%%%%%%%%%%%%%%%
%%%%%%%%%%%%%%%%%%%%%%%%%%%%%%%%%%%%%%%%%%%%%%%%%%%%%%
\begin{frame}{Basic Event-Based Algorithm}

\begin{figure}[h!]
\begin{center}
  \includegraphics[height=3 in,clip]{fig/MC-algorithm}
\end{center}
  \label{fig:mc-algo}
\end{figure}

\end{frame}

%%%%%%%%%%%%%%%%%%%%%%%%%%%%%%%%%%%%%%%%%%%%%%%%%%%%%%
%%%%%%%%%%%%%%%%%%%%%%%%%%%%%%%%%%%%%%%%%%%%%%%%%%%%%%
\begin{frame}{Let's Get Started With}

    \begin{enumerate}
    \item Physics as Probability
    \item Definitions: PDF \& CDF
    \item Motivation \& Goal of Random Sampling
    \item Basic Random Sampling Techniques
      \begin{itemize}
        \item Direct Discrete Sampling
        \item Direct Continuous Sampling
        \item Rejection Sampling
      \end{itemize}
    \end{enumerate}

\vspace*{1em}
Notes derived from Jasmina Vujic and Paul Wilson
\end{frame}



%%%%%%%%%%%%%%%%%%%%%%%%%%%%%%%%%%%%%%%%%%%%%%%%%%%%%%
%%%%%%%%%%%%%%%%%%%%%%%%%%%%%%%%%%%%%%%%%%%%%%%%%%%%%%
\begin{frame}{Learning Objectives}

    \begin{enumerate}
    \item Provide examples of probabilistic
representations of physics
    \item Distinguish between a PDF and CDF
    \item Distinguish between a \textit{discrete} PDF
(CDF) and \\ a \textit{continuous} PDF (CDF)
    \item Describe the goal of random sampling
    \item Identify and implement the best
random sampling technique for a given
distribution
    \end{enumerate}

\end{frame}


%%%%%%%%%%%%%%%%%%%%%%%%%%%%%%%%%%%%%%%%%%%%%%%%%%%%%%
%%%%%%%%%%%%%%%%%%%%%%%%%%%%%%%%%%%%%%%%%%%%%%%%%%%%%%
\begin{frame}{Physics as Probability}

Various physical phenomena can be
represented by probability distributions

\begin{itemize}
  \item Photon emission energy
    \begin{itemize}
    \item Each possible energy has a different probability
(intensity)
    \end{itemize} 
    
    \vspace*{0.5 em}
  \item Scattering cross-sections
    \begin{itemize}
    \item Each possible scattering angle has a different
probability as a function of the energy
    \end{itemize}
    
    \vspace*{0.5 em}
  \item Transmission through a medium
    \begin{itemize}
    \item Probability of reaching a particular position
depends on the cross-section
    \end{itemize}
\end{itemize}
\end{frame}


%%%%%%%%%%%%%%%%%%%%%%%%%%%%%%%%%%%%%%%%%%%%%%%%%%%%%%
%%%%%%%%%%%%%%%%%%%%%%%%%%%%%%%%%%%%%%%%%%%%%%%%%%%%%%
\begin{frame}{Probability Density Functions}

All variables, $x$, have a \underline{Probability Density Function (PDF)}, $p(x)$, with the following characteristics:

\begin{columns}
  \begin{column}{0.5\textwidth}
    \begin{center}
    \textcolor{berkeleygold}{\underline{Continuous}} 
    \end{center}
    %
    \begin{align*}
    p\left\lbrace a \leq x \leq b \right\rbrace &= \int_a^b p(x)dx\\
    & \\
    p(x) & \geq 0 \\
    \int_{-\infty}^{\infty} p(x)dx &= 1
    \end{align*}
  \end{column}
  \begin{column}{0.5\textwidth}
    \begin{center}
    \textcolor{berkeleyblue}{\underline{Discrete}}  
    \end{center}
    %
    \begin{align*}
    p(x = x_k) &= p_k \equiv p(x_k)\\
    k &= 1, \dots, N \\
    & \\
    p_k & \geq 0 \\
    \sum_{k=1}^N p_k &= 1
    \end{align*}
  \end{column}
\end{columns}

\end{frame}


%%%%%%%%%%%%%%%%%%%%%%%%%%%%%%%%%%%%%%%%%%%%%%%%%%%%%%
%%%%%%%%%%%%%%%%%%%%%%%%%%%%%%%%%%%%%%%%%%%%%%%%%%%%%%
\begin{frame}{Cumulative Distribution Functions}

All PDFs, $p(x)$, have an associated \underline{Cumulative Distribution Function (CDF)}, $P(x)$, with the following properties:

\begin{columns}
  \begin{column}{0.5\textwidth}
    \begin{center}
    \textcolor{berkeleygold}{\underline{Continuous}} 
    \end{center}
    %
    \begin{align*}
    P\left\lbrace x' \leq x \right\rbrace &= P(x) = \int_{-\infty}^x p(x')dx'\\
    & \\
    P(-\infty) &= 0 \:,\quad P(\infty) = 1 \\
    0 &\leq P(x) \leq 1 \\
    &\frac{dP(x)}{dx} \geq 0
    \end{align*}
  \end{column}
  \begin{column}{0.5\textwidth}
    \begin{center}
    \textcolor{berkeleyblue}{\underline{Discrete}}  
    \end{center}
    %
    \begin{align*}
    P\left\lbrace x' \leq x \right\rbrace &= P_k \equiv P(x_k) = \sum_{j=1}^k p_j\\
    k &= 1, \dots, N \\
    P_0 &= 0 \:,\quad P_N = 1 \\
    0 &\leq P_k \leq 1 \\
    P_k & \geq P_{k-1} \\
    \end{align*}
  \end{column}
\end{columns}

\end{frame}


%%%%%%%%%%%%%%%%%%%%%%%%%%%%%%%%%%%%%%%%%%%%%%%%%%%%%%
%%%%%%%%%%%%%%%%%%%%%%%%%%%%%%%%%%%%%%%%%%%%%%%%%%%%%%
\begin{frame}{Random Sampling Basics}

  	\begin{figure}
  	\begin{center}
  		\includegraphics[height=2.75in,clip]{fig/cont-pdf-cdf}
	\end{center}
  	\end{figure}

\end{frame}


%%%%%%%%%%%%%%%%%%%%%%%%%%%%%%%%%%%%%%%%%%%%%%%%%%%%%%
%%%%%%%%%%%%%%%%%%%%%%%%%%%%%%%%%%%%%%%%%%%%%%%%%%%%%%
\begin{frame}{Random Sampling Basics}

  	\begin{figure}
  	\begin{center}
  		\includegraphics[height=2.75in,clip]{fig/cont-pdf-cdf-2}
	\end{center}
  	\end{figure}

\end{frame}

%%%%%%%%%%%%%%%%%%%%%%%%%%%%%%%%%%%%%%%%%%%%%%%%%%%%%%
%%%%%%%%%%%%%%%%%%%%%%%%%%%%%%%%%%%%%%%%%%%%%%%%%%%%%%
\begin{frame}{Random Sampling Basics}

  	\begin{figure}
  	\begin{center}
  		\includegraphics[height=2.75in,clip]{fig/cont-pdf-cdf-3}
	\end{center}
  	\end{figure}

\end{frame}

%%%%%%%%%%%%%%%%%%%%%%%%%%%%%%%%%%%%%%%%%%%%%%%%%%%%%%
%%%%%%%%%%%%%%%%%%%%%%%%%%%%%%%%%%%%%%%%%%%%%%%%%%%%%%
\begin{frame}{Random Sampling Basics}

  	\begin{figure}
  	\begin{center}
  		\includegraphics[height=2.75in,clip]{fig/disc-pdf-cdf}
	\end{center}
  	\end{figure}

\end{frame}


%%%%%%%%%%%%%%%%%%%%%%%%%%%%%%%%%%%%%%%%%%%%%%%%%%%%%%
%%%%%%%%%%%%%%%%%%%%%%%%%%%%%%%%%%%%%%%%%%%%%%%%%%%%%%
\begin{frame}{Why Random Sampling}

Various physical phenomena can be
represented by probabilistic distributions

\begin{itemize}
    \item The known probability distribution
represents the \textit{collective} behavior
\vspace*{1em}
    \item We need to know the behavior at \textit{each}
single event
\vspace*{1em}
    \item We need to \underline{recreate} the collective behavior after \underline{many} events
\end{itemize}

\end{frame}


%%%%%%%%%%%%%%%%%%%%%%%%%%%%%%%%%%%%%%%%%%%%%%%%%%%%%%
%%%%%%%%%%%%%%%%%%%%%%%%%%%%%%%%%%%%%%%%%%%%%%%%%%%%%%
\begin{frame}{Random Sampling Purpose}

Use a random process to select a single
value with the following requirements

\begin{itemize}
    \item Each sample should be independent from
other samples
\vspace*{1em}
    \item The PDF formed from a large number of
samples should converge to the initial PDF
\vspace*{1em}
    \item Recover the full resolution of the initial PDF
\end{itemize}

\end{frame}


%%%%%%%%%%%%%%%%%%%%%%%%%%%%%%%%%%%%%%%%%%%%%%%%%%%%%%
%%%%%%%%%%%%%%%%%%%%%%%%%%%%%%%%%%%%%%%%%%%%%%%%%%%%%%
\begin{frame}{Sampling Techniques}

Random sampling uses \underline{uniformly distributed
random variables} to \alert{choose a value for a
variable} according to its probability density
function

    \begin{itemize}
    \item \textit{Basic} sampling techniques
      \begin{itemize}
      \item Direct discrete sampling
      \item Continuous discrete sampling
      \item Rejection sampling
      \end{itemize}
    
    \vspace*{.5em}
    \item \textit{Advanced }sampling techniques
      \begin{itemize}
      \item Histogram
      \item Piecewise linear
      \item Alias sampling
      \item Advanced continuous PDFs
      \end{itemize}
    \end{itemize}

\end{frame}


%%%%%%%%%%%%%%%%%%%%%%%%%%%%%%%%%%%%%%%%%%%%%%%%%%%%%%
%%%%%%%%%%%%%%%%%%%%%%%%%%%%%%%%%%%%%%%%%%%%%%%%%%%%%%
\begin{frame}{Uniformly-Distributed Random Variable}

    \begin{itemize}
    \item Standard notation
      \begin{itemize}
      \item Single random variable: $\xi$
      \item Pair of random variables: $(\xi, \eta)$
      \end{itemize}
    \vspace*{.5em}
    \item PDF for random variables:
    \end{itemize}

\begin{columns}
  \begin{column}{0.5\textwidth}
    %
    \begin{align*}
    p(\xi) = \begin{cases} 
     1 & 0 \leq \xi < 1 \\ 
     0 & \text{ otherwise} 
     \end{cases}
    \end{align*}
  \end{column}
  \begin{column}{0.5\textwidth}
  	\begin{figure}
  	\begin{center}
  		\includegraphics[height=1.5in,clip]{fig/unif-dist-var}
	\end{center}
  	\end{figure}
  \end{column}
\end{columns}

\end{frame}


%%%%%%%%%%%%%%%%%%%%%%%%%%%%%%%%%%%%%%%%%%%%%%%%%%%%%%
%%%%%%%%%%%%%%%%%%%%%%%%%%%%%%%%%%%%%%%%%%%%%%%%%%%%%%
\begin{frame}{Direct Discrete Sampling}

Sampling Procedure

    \begin{itemize}
    \item Generate $\xi$
    \item Determine $k$ such that $P_{k-1} \leq \xi \leq P_k$
    \item Return $x = x_k$
    \end{itemize}

  	\begin{figure}
  	\begin{center}
  		\includegraphics<1>[height=1.5in,clip]{fig/disc-pdf}
  		\includegraphics<2>[height=1.5in,clip]{fig/disc-pdf-samp}
	\end{center}
  	\end{figure}

\end{frame}


%%%%%%%%%%%%%%%%%%%%%%%%%%%%%%%%%%%%%%%%%%%%%%%%%%%%%%
%%%%%%%%%%%%%%%%%%%%%%%%%%%%%%%%%%%%%%%%%%%%%%%%%%%%%%
\begin{frame}{Direct Discrete Sampling}

\vspace*{1 em}
Consider the CDF

  	\begin{figure}
  	\begin{center}
  		\includegraphics[height=2.5in,clip]{fig/disc-cdf}
	\end{center}
  	\end{figure}

\end{frame}


%%%%%%%%%%%%%%%%%%%%%%%%%%%%%%%%%%%%%%%%%%%%%%%%%%%%%%
%%%%%%%%%%%%%%%%%%%%%%%%%%%%%%%%%%%%%%%%%%%%%%%%%%%%%%
\begin{frame}{Direct Discrete Sampling}

    \begin{itemize}
    \item Requires a table search on $P_k$
      \begin{itemize}
      \item Linear search requires $O(N)$ time
      \item Binary search requires $O(\log_2 N)$ time
      \end{itemize}
    \vspace*{.5em}
    
    \item Special case: Uniform discrete PDF
      \begin{itemize}
      \item $p_k = 1/N$
      \item $P_k = k/N$
      \item $k = \left\lfloor{1 + N\xi}\right\rfloor$ (floor function)
      \end{itemize}
    \end{itemize}

\end{frame}


%%%%%%%%%%%%%%%%%%%%%%%%%%%%%%%%%%%%%%%%%%%%%%%%%%%%%%
%%%%%%%%%%%%%%%%%%%%%%%%%%%%%%%%%%%%%%%%%%%%%%%%%%%%%%
\begin{frame}{Direct Continuous Sampling}

    \begin{itemize}
    \item Can only be used if CDF can be inverted
    \item Direct solution of $P(x)= \xi$
    \item Sampling Procedure:
    \end{itemize}
    \hspace*{3 em} Generate $\xi$ , \hspace*{1 em} Determine $x = P^{-1}(\xi)$
  	\begin{figure}
  	\begin{center}
  		\includegraphics<1>[height=1.5in,clip]{fig/cont-pdf}
  		\includegraphics<2>[height=1.5in,clip]{fig/cont-cdf-on-pdf}
  		\includegraphics<3>[height=1.5in,clip]{fig/cont-cdf-on-pdf-grey}
	\end{center}
  	\end{figure}    
    
\end{frame}


%%%%%%%%%%%%%%%%%%%%%%%%%%%%%%%%%%%%%%%%%%%%%%%%%%%%%%
%%%%%%%%%%%%%%%%%%%%%%%%%%%%%%%%%%%%%%%%%%%%%%%%%%%%%%
\begin{frame}{Direct Continuous Sampling}

    \begin{itemize}
    \item Advantages: 
      \begin{itemize}
      \item Straightforward math \& coding
      \end{itemize}
    \vspace*{1 em}
    \item Disadvantages:
      \begin{itemize}
      \item Can involve computationally slow functions
      \item Not always possible to invert $P(x)$
      \end{itemize}
    \end{itemize}
    
\end{frame}


%%%%%%%%%%%%%%%%%%%%%%%%%%%%%%%%%%%%%%%%%%%%%%%%%%%%%%
%%%%%%%%%%%%%%%%%%%%%%%%%%%%%%%%%%%%%%%%%%%%%%%%%%%%%%
\begin{frame}{Normalization}

    \begin{itemize}
    \item Random sampling depends on \textbf{shape} and not on \sout{magnitude}
    \item Normalization for formal definition of PDF/CDF
required
    \end{itemize}
%
\[
  \begin{aligned}
g(t)dt &= e^{-\lambda t} dt \:, \quad t > 0 \\
  %
G(t) &= \int_{-\infty}^t g(t')dt' = \int_0^t g(t') dt' = \biggl[- \frac{e^{-\lambda t'}}{\lambda} \biggr]_0^t = \frac{1}{\lambda} \bigl(1 - e^{-\lambda t}\bigr)\\
    G(\infty) &= \frac{1}{\lambda}\\
    %
\\p(t) &= \lambda g(t) = \lambda e^{-\lambda t} dt \:, \quad t > 0 \\
    %
    P(t) &= \int_{-\infty}^t p(t')dt' = \int_0^t \lambda f(t') dt' = \bigl[e^{-\lambda t'} \bigr]_0^t = 1 - e^{-\lambda t}\\
    %
    P(\infty) &= 1
  \end{aligned}
\]    
    
\end{frame}


%%%%%%%%%%%%%%%%%%%%%%%%%%%%%%%%%%%%%%%%%%%%%%%%%%%%%%
%%%%%%%%%%%%%%%%%%%%%%%%%%%%%%%%%%%%%%%%%%%%%%%%%%%%%%
\begin{frame}{Shifted Uniform}

\[
  \begin{aligned}
g(x) dx &= C dx \quad a \leq x < b\\
  %
G(x) &= \int_{-\infty}^x g(x')dx' = C \int_a^x dx' = C\bigl[x' \bigr]_a^x = C \bigl(x - a)\\
    G(\infty) &= G(b) = C(b-a)\\
    %
\\p(x) &= \frac{g(x)}{G(\infty)} = \frac{C}{C(b-a)} = \frac{1}{b-a} \quad a \leq x < b\\
    %
P(x) &= \int_{-\infty}^x p(x')dx' = \frac{1}{b-a}\int_a^x dx' = \frac{x-a}{b-a}\\
    %
    &\\
    x &= P^{-1}(\xi) = \xi\bigl(b-a\bigr) + a
  \end{aligned}
\]    
    
\end{frame}


%%%%%%%%%%%%%%%%%%%%%%%%%%%%%%%%%%%%%%%%%%%%%%%%%%%%%%
%%%%%%%%%%%%%%%%%%%%%%%%%%%%%%%%%%%%%%%%%%%%%%%%%%%%%%
\begin{frame}{Simple Line, Slope = $m$}

\[
  \begin{aligned}
g(x) dx &= mx\:dx \qquad 0 \leq x < 1\\
  %
G(x) &= \int_{-\infty}^x g(x')dx' = \int_0^x mx' dx' = \frac{m}{2}\bigl[x'^2 \bigr]_0^x = \frac{m}{2} x^2\\
    G(\infty) &= G(1) = \frac{m}{2}\\
    %
\\p(x) &= \frac{mx}{\frac{m}{2}} = 2x \qquad 0 \leq x < 1\\
    %
P(x) &= \int_{-\infty}^x p(x')dx' = \int_0^x 2x' dx' = \bigl[x'^2 \bigr]_0^x = x^2\\
    %
    &\\
    x &= P^{-1}(\xi) = \sqrt{\xi} \qquad\text{Independent of }m
  \end{aligned}
\]    
    
\end{frame}


%%%%%%%%%%%%%%%%%%%%%%%%%%%%%%%%%%%%%%%%%%%%%%%%%%%%%%
%%%%%%%%%%%%%%%%%%%%%%%%%%%%%%%%%%%%%%%%%%%%%%%%%%%%%%
\begin{frame}{Shifted Line}

\[
  \begin{aligned}
  g(x) dx &= m(x - a)\:dx \qquad a \leq x < b\\
  %
  G(x) &= \int_{-\infty}^x g(x')dx' = \int_a^x m(x'-a) dx' = \frac{m}{2}\bigl[(x'-a)^2 \bigr]_0^x = \frac{m}{2} (x-a)^2\\
    G(\infty) &= G(1) = \frac{m}{2}(b-a)^2\\
    %
    \\p(x) &= \frac{m(x-a)}{\frac{m}{2}(b-a)^2} = 2 \frac{x-a}{(b-a)^2}\qquad a \leq x < b\\
    %
    P(x) &= \int_{-\infty}^x p(x')dx' = \frac{1}{(b-a)^2}\int_a^x 2(x'-a) dx' = \frac{(x-a)^2}{(b-a)^2}\\
    %
    &\\
    x &= P^{-1}(\xi) = \sqrt{\xi}(b-a) + a \qquad\text{Independent of }m
  \end{aligned}
\]    
    
\end{frame}


%%%%%%%%%%%%%%%%%%%%%%%%%%%%%%%%%%%%%%%%%%%%%%%%%%%%%%
%%%%%%%%%%%%%%%%%%%%%%%%%%%%%%%%%%%%%%%%%%%%%%%%%%%%%%
\begin{frame}{Rejection Sampling}

    \begin{itemize}
    \item Many CDFs cannot be inverted
      \begin{itemize}
      \item e.g.\ Klien-Nishina cross-section
      \end{itemize}
    \vspace*{1 em}
    
    \item Use an approach that is more graphical
      \begin{itemize}
      \item Select a point in a 2-D domain
      \item Determine whether that point is above or below the PDF
      \item Keep those that are below
      \item Start over if above
      \end{itemize}
    \end{itemize}

\end{frame}


%%%%%%%%%%%%%%%%%%%%%%%%%%%%%%%%%%%%%%%%%%%%%%%%%%%%%%
%%%%%%%%%%%%%%%%%%%%%%%%%%%%%%%%%%%%%%%%%%%%%%%%%%%%%%
\begin{frame}{Rejection Sampling}

    \begin{itemize}
    \item Select a bounding function, g(x), such that
      \begin{itemize}
      \item $g(x)\geq p(x)$ for all $x$
      \item $g(x)$ is easy to sample
      \end{itemize}
    \item Simplest choice is $g(x) = C$
    \item May not be best choice
    \vspace*{1 em}
    
    \item Generate pair of random variables, $(\xi, \eta)$
      \begin{itemize}
      \item $x' = G^{-1} (\xi)$
      \item If $\eta < p(x')/g(x')$, accept $x'$
      \item Else, reject $x'$
      \end{itemize}
    \end{itemize}

\end{frame}

%%%%%%%%%%%%%%%%%%%%%%%%%%%%%%%%%%%%%%%%%%%%%%%%%%%%%%
%%%%%%%%%%%%%%%%%%%%%%%%%%%%%%%%%%%%%%%%%%%%%%%%%%%%%%
\begin{frame}{Rejection Sampling}

  	\begin{figure}
  	\begin{center}
  		\includegraphics<1>[height=2.5in,clip]{fig/rej-samp1}
  		\includegraphics<2>[height=2.5in,clip]{fig/rej-samp2}
  		\includegraphics<3>[height=2.5in,clip]{fig/rej-samp3}
  		\includegraphics<4>[height=2.5in,clip]{fig/rej-samp4}
	\end{center}
  	\end{figure} 

\end{frame}


%%%%%%%%%%%%%%%%%%%%%%%%%%%%%%%%%%%%%%%%%%%%%%%%%%%%%%
%%%%%%%%%%%%%%%%%%%%%%%%%%%%%%%%%%%%%%%%%%%%%%%%%%%%%%
\begin{frame}{Rejection Sampling}

    \begin{itemize}
    \item Advantages
      \begin{itemize}
      \item Computationally simple
      \item Always works
      \end{itemize}
    \vspace*{1 em}
    
    \item Disadvantages
      \begin{itemize}
      \item Will be inefficient if shapes of $g(x)$ and $p(x)$ are not similar
      \end{itemize}
    \end{itemize}
    
    \begin{equation}
    \text{Efficiency }= \frac{\int p(x)dx}{\int g(x) dx} \nonumber
    \end{equation}

\end{frame}


%%%%%%%%%%%%%%%%%%%%%%%%%%%%%%%%%%%%%%%%%%%%%%%%%%%%%%
%%%%%%%%%%%%%%%%%%%%%%%%%%%%%%%%%%%%%%%%%%%%%%%%%%%%%%
\begin{frame}{Random Sampling Summary}

    \begin{itemize}
    \item Physics can be represented \textit{probabilistically}

\vspace*{.5em}
    \item We can create PDFs and from those generate CDFs

\vspace*{.5em}
    \item These can be either continuous or discrete

\vspace*{.5em}
    \item We learned some basic ways to use random numbers to \textit{sample} from these distributions to \alert{simulate physics}
    \end{itemize}
    
\end{frame}

\end{document}
